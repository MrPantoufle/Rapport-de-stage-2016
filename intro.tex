
\section{Introduction}
It is often said in complexity theory that polynomial-time 
solvable problems are "easy" and that NP-hard problems are always
"hard" to solve. However this has been challenged in recent
years, by the theory of \emph{fine-grained complexity} \cite{borassi2015into,
vassilevskawilliams:LIPIcs:2015:5568}.
Indeed, a graph algorithm running in quadratic time on a 
million-node network may be considered too slow ;
in the opposite direction, it is still not proved that the $2^n$ bruteforce
 algorithm
for SAT is optimal, under the assumption that $P \neq NP$.

\todo[inline]{blablabla}

Our focus will be primarily on boolean matrix multiplication algorithms, and
other graph-related problems. We try to grasp the notion of \emph{combinatorial}
algorithm, that is, an algorithm that does not rely on infinite precision
arithmetic (such as \emph{classic} algorithms for fast matrix 
multiplication). We also mark a difference between \emph{dense} and 
\emph{sparse} data on which we run our algorithms: representing a graph
(or equivalently a boolean matrix) by an adjacency list can be an 
essential step in making a problem tractable, and thus deserves special
attention (in particular, these algorithms can often be 
output-sensitive).

We present our main result
\begin{Theo}
\label{quick_trans}
Algorithm \ref{transitive_closure_alg_optim} computes the transitive closure
of a graph in $\BigO(n+m^{3/2})$ time, with $m$ being the number of edges
in the transitive closure.
\end{Theo}
To show this we build up on a recent analysis of a classic algorithm for
computing the 
transitive closure of a graph, and manage a logarithmic gain in  time while
 using linear space.

\begin{Def}
A linear reduction from a problem $\mathcal P$ to a problem
$\mathcal Q$ is a function $\Phi$ from instances of $\mathcal P$
to instances of $\mathcal Q$ satisfying, for every instance $I$ of
$\mathcal P$, the following two properties.
\begin{itemize}
\item $\Phi(I)$ can be computed in time $\BigO(s(I))$, where $s(I)$
is the size of input $I$.
\item $I$ and $\Phi(I)$ have the same output (if the ouput is not boolean,
we will also require a linear-time computable function that transforms
the output of $\Phi(I)$ into the output of $I$).
\end{itemize}
If there exists a linear time reduction from $\mathcal P$ to
$\mathcal Q$, we will write $\mathcal P \Lred \mathcal Q$.
\end{Def}


\begin{Def}
A quasilinear Karp reduction from a problem $\mathcal P$ to a problem
$\mathcal Q$ is a function $\Phi$ from instances of $\mathcal P$
to instances of $\mathcal Q$ satisfying, for every instance $I$ of
$\mathcal P$, the following two properties.
\begin{itemize}
\item $\Phi(I)$ can be computed in time $\tilde\BigO(s(I))$, where $s(I)$
is the size of input $I$.\footnote{By $\tilde\BigO(f(n))$ we mean 
$\BigO(f(n)\log^k n)$ for some fixed $k$.}
\item $I$ and $\Phi(I)$ have the same output (if the ouput is not boolean,
we will also require a linear-time computable function that transforms
the output of $\Phi(I)$ into the output of $I$).
\end{itemize}
If there exists a quasilinear time reduction from $\mathcal P$ to
$\mathcal Q$, we will write $\mathcal P \Qlred \mathcal Q$.
\end{Def}

\subsection{Some problems related to matrix multiplication}
In \cite{kratsch2006between}, Kratsch and Spinrad establish a series
of reductions to three problems they consider more fundamental:
\medskip

\begin{tabular}{|rl|}
\hline
\multirow{5}{*}{\Prob{TriangleFree} $\Lred$} 
%& \multirow{5}{*}{$\Lred$} \\
& \Prob{ATFreeGraph} \\
& \Prob{GraphDominatingPair} \\
& \Prob{DominatingPair} \\
& \Prob{GraphExtremities} \\
& \Prob{\#SimplicialVertices} \\ \hline
\multirow{2}{*}{\Prob{GraphDominatedVertex} $\Lred$}
%&& \multirow{2}{*}{$\Lred$}
& \Prob{GraphTwoPair} \\
& \Prob{GraphStarCutset} \\ \hline
\Prob{GraphSimplicialVertex} $\Lred$ & 
\Prob{GraphCliqueCutset} \\ \hline
\end{tabular}
